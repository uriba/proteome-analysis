\documentclass[10pt,letterpaper]{article}
\usepackage{amsmath,amssymb}
\usepackage{cite}
\usepackage{nameref,hyperref}
% Use the PLoS provided BiBTeX style
\bibliographystyle{plos2015}

% Remove brackets from numbering in List of References
\makeatletter
\renewcommand{\@biblabel}[1]{\quad#1.}
\makeatother

% Leave date blank
\date{}

\usepackage{latexsym}
\usepackage{amsfonts}
\usepackage{pgfplotstable}
\usepackage{url}
\usepackage[nomessages]{fp}
\newcommand{\hConds}{$20$}
\newcommand{\hCondsLBnum}{22}
\newcommand{\hCondsLB}{$\hCondsLBnum$}
\newcommand{\voConds}{$5$}
\newcommand{\vnCondsnum}{23}
\newcommand{\vnConds}{$\vnCondsnum$}
\newcommand{\vConds}{\vnConds{}}
\FPeval{\totConds}{clip(\hCondsLBnum+\vnCondsnum+2)}
\FPeval{\tothConds}{clip(\hCondsLBnum+2)}
\newcommand{\hTotal}{$1442$}
\newcommand{\hGlobal}{$473$}
\newcommand{\hGlobalLB}{$352$}
\newcommand{\hGlobalShuff}{$12$}

\newcommand{\hHorizIntercept}{$-0.36$}
\newcommand{\hHalflife}{$1.9$}

\newcommand{\vnTotal}{$1142$}
\newcommand{\voTotal}{$919$}
\newcommand{\vTotal}{\vnTotal{}}

\newcommand{\voGlobalShuff}{$152$}
\newcommand{\vnGlobalShuff}{$5$}
\newcommand{\vGlobalShuff}{\vnGlobalShuff{}}

\newcommand{\vnGlobal}{$305$}
\newcommand{\voGlobal}{$378$}
\newcommand{\vGlobal}{\vnGlobal{}}

\newcommand{\vnIntercept}{$-0.54$}
\newcommand{\voIntercept}{$-0.54$}
\newcommand{\vHalflife}{$1.29$}
\newcommand{\hProtsInRange}{$414$}
\newcommand{\vProtsInRange}{$221$}

\newcommand{\hMeanSlope}{$1.32$}
\newcommand{\hMeanSlopeStdDev}{$0.57$}
\newcommand{\hSlopeStdErrMean}{$0.34$}
\newcommand{\hRibsMeanSlope}{$1.41$}
\newcommand{\hRibsSlopeStdDev}{$0.32$}
\newcommand{\hRibsSlopeStdErrMean}{$0.23$}
\newcommand{\hGlobalSumSlope}{$1.24$}
\newcommand{\hGlobalSumRsq}{0.91}
\newcommand{\hRibsSumSlope}{$1.37$}
\newcommand{\hRibsSumRsq}{0.89}

\newcommand{\hRibs}{$53$}
\newcommand{\hCorrRibs}{$47$}

\newcommand{\vnRibs}{$53$}
\newcommand{\voRibs}{$54$}
\newcommand{\vRibs}{\vnRibs{}}
\newcommand{\vnCorrRibs}{$52$}
\newcommand{\voCorrRibs}{$52$}
\newcommand{\vCorrRibs}{\vnCorrRibs{}}

\newcommand{\vnMeanSlope}{$1.06$}
\newcommand{\voMeanSlope}{$1.39$}
\newcommand{\vMeanSlope}{\vnMeanSlope{}}

\newcommand{\vMeanSlopeStdDev}{$0.60$}
\newcommand{\vSlopeStdErrMean}{$0.14$}
\newcommand{\vRibsMeanSlope}{$1.48$}
\newcommand{\vRibsSlopeStdDev}{$0.08$}
\newcommand{\vRibsSlopeStdErrMean}{$0.06$}

\newcommand{\vnGlobalSumSlope}{$1.0$}
\newcommand{\voGlobalSumSlope}{$1.39$}
\newcommand{\vGlobalSumSlope}{\vnGlobalSumSlope{}}

\newcommand{\vnGlobalSumRsq}{0.97}
\newcommand{\voGlobalSumRsq}{1.0}
\newcommand{\vGlobalSumRsq}{\vnGlobalSumRsq{}}

\newcommand{\vnRibsSumSlope}{$1.49$}
\newcommand{\voRibsSumSlope}{$1.49$}
\newcommand{\vRibsSumSlope}{\vnRibsSumSlope{}}

\newcommand{\vnRibsSumRsq}{0.97}
\newcommand{\voRibsSumRsq}{0.97}
\newcommand{\vRibsSumRsq}{\vnRibsSumRsq{}}

\newcommand{\hMaxExpVar}{0.082}
\newcommand{\vnMaxExpVar}{0.035}
\newcommand{\voMaxExpVar}{0.04}
\newcommand{\vMaxExpVar}{\vnMaxExpVar{}}

\begin{document}
\paragraph{Effects of degradation and varying synthesis rate on model predictions}
The predicted fraction of an unregulated protein as a function of the growth rate is to follow a linear trend crossing at the origin (S1 Fig, blue dots).
Degradation can be interpreted as implying that the observed growth rate is the combination of the bio-synthesis rate minus the degradation rate, implying that the predicted fraction of an unregulated protein is linear increase with growth rate, but with a horizontal intercept at minus the degradation rate as is shown by the green dots in S1 Fig.

Non constant bio-synthesis rate can be modeled as a Michaelis-Menten kinetic like interdependence with growth rate following the formula:
\begin{equation}
\eta(g(c))=\frac{\eta_0}{1+\frac{g_m}{g(c)}}
\end{equation}
where $g(c)$ is the growth rate, $\eta(g(c))$ is the bio-synthesis rate at growth rate $g(c)$, $\eta_0$ is the maximal bio-synthesis rate and $g_m$ the growth rate at which the bio-synthesis rate is $\frac{1}{2}$ the maximal rate (S1 Fig, red line).
Under this assumption, the doubling time of the bio-synthesis machinery itself, $T_B$ becomes:
\begin{equation}
\label{eq:varrate}
T_B=T_B^0\frac{\eta(g(c))}{\eta_0}=T_B^0(1+\frac{g_m}{g(c)})
\end{equation}
where $T_B^0$ is the minimal theoretical doubling time when all the proteins are bio-synthesis proteins operating at maximal rate.
Substituting Eq.(\ref{eq:varrate}) into Eq.(\ref{eq:default-response}) results in a predicted fraction of:
\begin{equation}
p_i(c)=\frac{w_i(c)}{W_B}\frac{T_B^0(1+\frac{g_m}{g(c)})}{\ln(2)}g(c) = \frac{w_i(c)}{W_B}\frac{T_B^0}{\ln(2)}(g(c)+g_m)
\end{equation}
Surprisingly, this equation also describes the fraction as being linearly dependent on the growth rate, with the kinetic parameters implying a non-zero fraction at zero growth rate as is shown by the red dots in S1 Fig.
\end{document}
